\documentclass[review]{elsarticle}

\usepackage{lineno,hyperref}
\modulolinenumbers[5]
\hypersetup{
    colorlinks=true,
    linkcolor=blue,
    filecolor=magenta,      
    urlcolor=cyan,
    pdftitle={Overleaf Example},
    pdfpagemode=FullScreen,
}
\usepackage{amsmath}

\journal{{\tt no Journal}}

%%%%%%%%%%%%%%%%%%%%%%%
%% Elsevier bibliography styles
%%%%%%%%%%%%%%%%%%%%%%%
%% To change the style, put a % in front of the second line of the current style and
%% remove the % from the second line of the style you would like to use.
%%%%%%%%%%%%%%%%%%%%%%%

%% Numbered
%\bibliographystyle{model1-num-names}

%% Numbered without titles
%\bibliographystyle{model1a-num-names}

%% Harvard
%\bibliographystyle{model2-names.bst}\biboptions{authoryear}

%% Vancouver numbered
%\usepackage{numcompress}\bibliographystyle{model3-num-names}

%% Vancouver name/year
%\usepackage{numcompress}\bibliographystyle{model4-names}\biboptions{authoryear}

%% APA style
%\bibliographystyle{model5-names}\biboptions{authoryear}

%% AMA style
%\usepackage{numcompress}\bibliographystyle{model6-num-names}

%% `Elsevier LaTeX' style
\bibliographystyle{elsarticle-num}
%%%%%%%%%%%%%%%%%%%%%%%

\begin{document}

\begin{frontmatter}

\title{Estimating the number of cases of {COVID-19} from population isolation level}
%\tnotetext[mytitlenote]{Source code and data are available in the Github repository \href{https://github.com/aholanda/pandiso}{pandiso}.}

%% or include affiliations in footnotes:
\author[mymainaddress,mysecondaryaddress]{Adriano J.\ Holanda}
\ead{aholanda@usp.br}

\address[mymainaddress]{Department of Computing and Mathematics - Faculty of Philosophy, Science and Letters at Ribeir\~{a}o Preto - University of S\~{a}o Paulo, Ribeirão Preto, S\~{a}o Paulo, Brazil}
\address[mysecondaryaddress]{Faculty ``Dr.\ Francisco Maeda'', Ituverava, São Paulo, Brazil}

%\section{Front matter}

\begin{abstract}
We use the logistic function to estimate the number 
of individuals infected by a virus in a period of time
using as parameter the population average isolation level 
in the previous period of the infection occurrences.
Each period is composed by a date range in days 
where the isolation level is supposed to take
effect over the virus spread over the next 
date range.
The sample is the \hbox{COVID-19} cases and 
isolation level data from S\~{a}o Paulo State, Brazil.  
The proposed method is divided into two phases: 
1) The logistic function is fitted against \hbox{COVID-19}
empirical data to obtain the function parameters; 
2) the fitted paramaters, 
with the exception of overall growth factor,
 and the average isolation level for all periods of time 
are used to calculate a constant called $\lambda$ 
that is used to obtain the logistic growth factor and
 isolation level at each time step.
 The number of cases is estimated using 
logistic function and the growth factor 
from previous period of time 
to obtain the effect of people isolation during the
elapsed time. 
The period of time that produces a 
better correlation between empirical and estimated 
number of cases is 5 days. 
We conclude the method produces a good 
estimation when correlated with the empirical data.
\end{abstract}

\begin{keyword}
\hbox{SARS-CoV-2}\sep \hbox{COVID-19}\sep logistic function \sep isolation
\end{keyword}

\end{frontmatter}

\linenumbers

\section{Introduction}

{SARS-CoV-2} is the virus that causes the infection 
called coronavirus disease (\hbox{COVID-19}). 
COVID-19 pandemic control will depend 
on the coverage of population immunity obtained 
through infection or vaccination~\cite{WHOa}. 
The mutants of SARS-Cov-2 impose a challenge in 
the predictability in the long-term efficacy 
of the current vaccines. 
 It's of utmost importance the preparation of updated 
 vaccines tailored to emerging variants that are cross-reactive 
 against all circulating variants~\cite{Harvey2021}. 
Indeed, a small percentage of population in some countries 
is immunized against SARS-CoV-2, 
e.g., only 13.7\% of the population                                 % TODO: update percentage
is fully vaccinated in Brazil~\cite{MH2021}.
The {SARS-Cov-2} variants and its low rate of immunization 
impose the continuation of non-pharmacological approaches: 
handwashing, use of facial masks and social distancing~\cite{WHOb}. 
Among these approaches, only social distancing 
can be assessed effectively using 
communication traces~\cite{Farrahi2014}. 
 Isolating cases, quarantining contacts and implementing
large-scale social distancing approaches have been proving to 
be effective in the control of virus spread~\cite{Aquino2020}.

In this study, we use the logistic function 
to calculate the expected number of {COVID-19} cases 
as a function of time
based 
on the isolation level of the population and 
observed number of cases. 
We use S\~{a}o Paulo State, Brazil data as sample, 
but the method is simple and generic 
to apply to any population data. 
S\~{a}o Paulo is one and the most populous of the 26 states 
of the Federative Republic of Brazil with 
more than 40-million inhabitants, 21.9\% 
of Brazilian population and is responsible for 33.9\% 
of Brazil's GPD (Gross Domestic Product) 
(\url{https://www.ibge.gov.br/cidades-e-estados/sp.html}).

\section{Methods}

\subsection{Data}

We gathered the number of cases of COVID-19 in S\~{a}o Paulo State, Brazil, from Github 
repository of ``S\~{a}o Paulo Plan'' (\url{https://github.com/seade-R/dados-covid-sp}) 
that is a initiative from the state government to organize and process data from COVID-19 
from all cities of the state. The aim is to help in the decision making to implement public 
procedures to control the SARS-Cov-2 spread. We fetch the population isolation level data 
for all state cities also from 
``S\~{a}o Paulo Plan'', but from another site (\url{https://www.saopaulo.sp.gov.br/coronavirus/isolamento/}).

We've developed a script using R programming language to perform the sum of the 
number of cases and to calculate the average isolation level for all cities per day and to save
 the results in a separate file. We've used the script to process the daily assessment using
 the date steps (days) as an argument to group the data. In each group of days,
 the cases are summed and the isolation levels averaged.

\subsection{Fitting}

We choose the logistic function to model the COVID-19 spread behavior. 
The plot the cumulative number of cases of COVID-19 at each time step 
was performed and the data was fitted using the solution for the logistic function:

\begin{equation}
{dN \over dt} = r N (1 - {N \over K})
\label{eq:df:logistic}
\end{equation}

\noindent that is

\begin{equation}
N(t) = {K \over {1- {(K-N_{m}) \over N} e^{-rt}}}
\label{eq:exp:logistic}
\end{equation}

\noindent where $K$ is the maximum value reachable by $N$, 
$N_m$ is the curve midpoint, 
$r$ is the growth rate 
and $t$ are the time steps.

Figure \ref{fig:logistic} shows a hypothetical logistic function 
plotted using equation \ref{eq:df:logistic} where before the midpoint 
the differential normalized number of cases between 
two consecutives time steps increases, 
and after the midpoint decreases. 
The curve reaches a plateau in the maximum value for $n(t)$. 
In an empirical scenario, the plateau could be achieved 
if the process that is feeding the curve 
growth is controlled with the maximum 
value for $n(t)$ not being reached.

\begin{figure}
\includegraphics[scale=.625]{logistic.png}
\caption{Hypothetical normalized logistic function -- 
curve of equation \ref{eq:df:logistic} with 
$K=1.0$ and an arbitrary growth rate $r=0.04$; 
the midpoint coordinates are $(0.5, 0.5)$.
The $K$ represents a normalized measure where 
the maximum value is the divisor factor.}
\label{fig:logistic}
\end{figure}

We considered $N$ to be the number of infected individuals 
and choose the logistic function because of its behavior, 
with an exponential increase at the beginning but slowing down after 
a  midpoint and reaching a plateau. 
The same behavior occurs with a virus dissemination when at the 
beginning it spreads exponentially and it starts to 
decrease the contamination after some control protocol is implemented, 
like vaccination or lockdown, 
or in the worst case scenario when the majority of the population is infected.

\subsection{Estimation}

We postulate that the growth rate $r$ in the logistic function is 
inversely proportional to the population isolation level $i$ 
modulated by a constant $\lambda$:

\begin{equation}
r_{\text{fit}} = {\lambda /\, \overline{i\,}} \Rightarrow \lambda = r_{\text{fit}}\, \overline{i\,}
\label{eq:lambda}
\end{equation}

We calculate the constant by multiplying $r_{\text{fit}}$ 
obtained with the fitting of 
logistic function and the empirical data by 
the average isolation level $\overline{i\,}$  
of all time steps.

The constant calculated using equation \ref{eq:lambda} 
and isolation level of each time interval $i(t)$ 
are used to estimate the growth $r(t)$ using the formula:

\begin{equation}
r(t) = {\lambda\over i(t)}
\label{eq:rt}
\end{equation}

A time step $\Delta t$ represents the period of time used to group the data 
and the period in which the isolation level takes effect over
 the people's contacts, consequently affecting the SARS-CoV-2 spread. 
The estimated number of cases $N(t+\Delta t)$ is calculated using
 equation \ref{eq:exp:logistic}, using $r(t)$ instead of $r(t+\Delta t)$ since 
the isolation level is inversely proportional to growth rate, we take
 the previous growth rate to predict the effect in the number of cases after a 
period of time $\Delta t$:

\begin{equation}
N(t+\Delta t) = {K \over {1- {(K-N_{m}) \over N} e^{-rt}}}
\label{eq:exp:delta:logistic}
\end{equation}

 We compare these results with the empirical data using Pearson correlation.
 We use a time step $t$ 
 equals to $5$ days because it was the value with
 a better correlation between empirical and estimated data. 
We only use integers as time intervals for days because
 the variation in the correlation values was not
 significant to justify handling fractional 
values  that increases the complexity
 of code and analysis.

\section{Results}

Figure \ref{fig:fitting} shows the average number of cases per 
time step in São Paulo State due COVID-19 and 
the fitting performed using logistic function (equation \ref{eq:exp:delta:logistic}) 
that proved to be a good approximation to the 
cumulative growth of number of cases.

It’s striking to notice the consistent increase in the number of cases 
with the plateau to be achieved when the number of cases reaches $1,105,681$ 
according to the parameters extracted from fitting.

\begin{figure}
\includegraphics[scale=.625]{fitting.png}
\caption{Curve of accumulated number of cases 
at each time step $t$ and its fitting curve 
using the logistic function (equation \ref{eq:exp:logistic}) 
where the parameters obtained from fitting are: 
$r=0.044, N_m=27,483, K=1,105,681$. 
The Pearson correlation between the empirical 
and fitted data is $0.994$ and 
its {\it p-value\/} is approximately $3\times 10^{-91}$.  
Other important properties are: 
number of time $steps=96$, 
initial $t_{0}="2020/03/01"$ 
and final $t_{f}="2020/06/22"$ 
dates.}
\label{fig:fitting}
\end{figure}

Figure \ref{fig:estimation} shows the empirical curve again 
and the estimated number of cases using 
the fitting parameters K and Nm 
and the calculation of growth factor $r(t)$ using equation \ref{eq:rt}, 
where the isolation level $i(t)$ is used, 
to plug into logistic function (equation \ref{eq:exp:delta:logistic}) 
and calculate the estimated number of cases $N(t+\Delta t)$.

FIXME: put a and b in the plot
\begin{figure}
\centering
\includegraphics[scale=.625]{estimation.png}
\caption{a) Curve of accumulated number of cases 
and the estimation using the logistic 
function~(equation \ref{eq:exp:delta:logistic}) 
with the growth factor $r(t)$ obtained from equation~\ref{eq:rt}. 
The constant $\lambda$ was calculated using equation~\ref{eq:lambda} with 
$r=0.044$ and \hbox{$\overline{i\,} = 0.438\pm 0.04$}. 
The Pearson correlation coefficient between empirical and estimated data is $0.99$ 
and its {\it p-value\/} is approximately \hbox{$2.13\times 10^{-81}$}. 
b) Average isolation level at each time step $t$. 
The number of steps, initial and final dates are 
the same as Figure \ref{fig:fitting}.}
\label{fig:estimation}
\end{figure}

We noticed at the beginning of the curve in Figure \ref{fig:estimation}b, high values of isolation levels. 
At that moment, the São Paulo State government adopted more strict rules to avoid circulation of people. 
It was issued from March 16th and March 22nd a decree to encourage home-based working whenever possible, 
closure of schools and non-emergencies commerce\cite{Cruz2020}.

\section{Discussion}

The proposed method to predict the number of infected individuals 
using the current empirical data and the logistic function proved 
to be an effective approximation for the infection spread. 
It’s a simple method and the fact it doesn’t take into account 
other barriers to the infection spread, like the use of a facial mask, 
is a good property since it’s very difficult to obtain data about 
the effect of these important procedures in the number of cases.

The main reason to develop a simple but accurate method to estimate 
the number of infected individuals using the isolation level is to help 
in the decision making process related to the control of infection spread. 
The isolation level is a reliable measure and may be used to predict 
the desired behavior of the population in time steps and check if the infection 
is really being controlled, reaching a remission in the near future.

We adopted a simpler method instead of more detailed 
compartmental models~\cite{Ross1916,Ross1917a,Ross1917b}, 
e.g. SIR (Susceptible, Infectious, or Recovered), SEIR (Susceptible, Exposed, Infectious, or Recovered) 
or SEIS (Susceptible, Exposed, Infectious, then Susceptible again), due the following reasons: 
a) The number of infected individuals is very small when compared with population if there is 
    an application of approaches to control the virus spread; 
b) It’s difficult to assess the influence of non-pharmacological approaches and the approximation 
    used somewhat seems to embed this influence due the high correlation 
   the empirical and estimated data; 
c) Birth, death, death due the disease, immigration and emigration rates 
   are also difficult to measure effectively, even deaths due to the disease 
   may be contaminated by errors due death certificate misjudgment; 
d) It’s challenging to assess, in some of these models, the number of 
    individuals with repeated infections and the gradual loss of acquired 
    immunity because some aspects of the immunization process is 
   yet to be unraveled, at least for COVID-19.

Another advantage of our method over the compartmental models is the 
 reduction of dimensionality to one manageable and reliable dimension, 
this process is also called feature engineering~\cite{Spieg2019}. 
This property also facilitates code understanding, maintenance and efficiency. 
The data is easily updated and the number of days in each time step 
may be changed in the source code.

For future developments, we visualize the use of Control Theory 
in the estimation phase~\cite{Stewart2020}. 
The isolation level and number of cases 
may be used as process variable and set point, respectively, 
in a controller like PI (Proportional-Integral) or PID (PI-Derivative), 
optimizing the system according to some properties. 
The properties may be the vaccination rate and the vaccines’ stock, 
number of ICU (Intensive Care Unit) beds in a hospital, 
and even immunization loss growth rate.

TODO: Search isolation papers to compare
TODO: read more control theory+covid papers to justify future works
TODO: Distinguish statistics population from state population

\section{Source code and data availability}
Source code and data required to reproduce the results 
obtained in this study are available at
\url{https://github.com/aholanda/pandiso}.

\bibliography{main}

\end{document}